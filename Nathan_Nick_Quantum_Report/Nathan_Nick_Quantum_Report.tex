\documentclass[conference]{IEEEtran}
%\IEEEoverridecommandlockouts
% The preceding line is only needed to identify funding in the first footnote. If that is unneeded, please comment it out.
\usepackage{algorithmic}
\usepackage{amsmath,amssymb,amsfonts}
\usepackage{cite}
\usepackage{graphicx}
\usepackage{listings}
\usepackage{pgfplots}
\pgfplotsset{width=7cm,compat=1.8}
\usepackage{textcomp}
\usepackage{xcolor}
\def\BibTeX{{\rm B\kern-.05em{\sc i\kern-.025em b}\kern-.08em
    T\kern-.1667em\lower.7ex\hbox{E}\kern-.125emX}}
\begin{document}

\title{Simulating Attacks on Authenticated Semi-Quantum Direct Communication Protocol}

\author{\IEEEauthorblockN{Nathan Marshall}
\IEEEauthorblockA{
Ottawa, Canada \\
nathsmar@gmail.com}
\and
\IEEEauthorblockN{Nick Rivard}
\IEEEauthorblockA{
Ottawa, Canada \\
nhrivard@gmail.com}
}

\maketitle

\begin{abstract}

This study presents a detailed analysis of the randomization-based authenticated semi-quantum direct
communication protocol presented by Yi-Ping Luo and Tzonelih Hwang, as well as several of the attacks
already examined in their analysis of the protocol. The protocol uses pre-shared keys
rather than a classical channel, and requires only the sender to have advanced quantum capabilities.
We have implemented a simulation of the protocol and confirmed that it is resistent to impersonation
attacks, intercept-and-resend attacks, and one-qubit modification attacks. We have also solved
the problem of reordering qubits into a random permutation based on a key,
which was not explained in the original description of the protocol.

\end{abstract}

\begin{IEEEkeywords}

Authentication,
Authenticated semi-quantum communication,
Bell states,
Quantum communication,
Quantum cryptography,
Semi-quantum communication

\end{IEEEkeywords}

\section{Introduction}

While practical quantum communication is feasible using
modern technology, the cost of nodes capable of quantum computation
will likely remain high for some time. Since simpler, "classical" nodes
(with only the ability to read individual qubits) are cheaper,
semi-quantum communication protocols are useful for practical,
cost-efficient, secure communication.





Several papers have tried to address this need, including
the paper the present work is based on: Authenticated semi-quantum
direct communication protocols using Bell states. The present work
describes that paper and the insights, results, and conclusions gained
from implementing one of two protocols described therein: Randomization-based
Authenticated Semi-Quantum Direct Communication (ASQDC) protocol.

The problem addressed in the present work was the implementation of
the above-mentioned protocol in Matlab using Quantum Information Toolkit
given several oversights in the specification presented in the original
paper which are described later in the current work. Implementing
the ASQDC protocol is relevant to the goal of realizing quantum communications
on a large scale because the protocol addresses several of the major
hurdles to widespread adoption of quantum communications while also
hand-waving certain considerations which, as demonstrated later in
the present work, require consideration by anyone who would use the
protocol or provide it as a service. Given the widely acknowledged
benefits of quantum communications, it therefore makes sense to explore
the practicalities and compromises that arise in implementation.

The authors successfully implemented the randomization-based ASQDC
protocol as per the specifications laid out in the original paper,
including generating data concerning the security of the protocol
in the presence of four attacks: impersonation of Alice, impersonation
of Bob, intercept-and-resend attack, and modification of 1-qubit attack.
These results are consistent with the analysis presented in the original
paper and support the claim that the protocol is secure against these
attacks, even for small message sizes which is the worst-case. Some
aspects of the security model did, however, need to be modified in
order to keep the implementation as faithful to the original specification
as possible.

\section{Problem Domain}

\subsection{Authentication}

Authentication - or, more specifically message authentication - is
the process of confirming the data origin integrity and the data integrity
of a given message. This means confirming both that the message did
in fact originate from a given sender and that the message was not
modified during transit. This is often done using a combination of
shared secrets (keys) and cryptographic hash functions. If the message
can only be transformed into its corresponding plaintext by a key
corresponding in some way to a key which is only in the possession
of a known sender then one can infer that the message must have been
sent by the known sender since it must have been encrypted using that
key. Similarly for data integrity, if a given plaintext is inputted
into a one-way function (a cryptographic hash function) and produces
the observed hash value, then it can be inferred that it is the same
message that was originally used by the known sender to produce the
hash before they encrypted the message. Of course, in reality there
can be many complications to the above, but this model is sufficient
for the present discussion.

\subsection{Requirements for Authentication in Traditional Protocols}

Typically, quantum communications protocols will implement message
authentication indirectly using authenticated classical (non-quantum)
channels and nodes which are capable of performing quantum computation.
This suggests restrictions on viable use-cases for message authentication
in quantum communications. First, a dedicated and highly available
authenticated classical channel might not be feasible in nodes are
battery powered. Second, while quantum computers exist now, they tend
to be expensive and sensitive to environmental conditions. This implies
that it will be difficult to build and deploy reliable mobile quantum
computers.

\subsection{Authenticated Semi-Quantum Key Distribution}

To address the requirement that all nodes be capable of quantum computation,
Yu et al. proposed Authenticated Semi-Quantum Key Distribution (ASQKD)
protocols. In these protocols, in order to circumvent the requirement
for an authenticated classical channel, keys are shared between sender
and receiver in advance. Furthermore, at the cost of making quantum
communications one-way, the receiver need not be capable of quantum
computation themselves. To be precise, Bob (the receiver) only needs
to be able to perform the following operations: (1) prepare qubits
in the classical basis $\left\{ \left|0\right\rangle ,\left|1\right\rangle \right\} $,
(2) measure individual qubits in the the classical basis, (3) permute
qubits, and (4) reflect qubits without measuring them. This is in
contrast to Alice (the sender), who need to be able to perform the
following operations: (1) prepare any arbitrary quantum states and
(2) perform arbitrary quantum measurement.

\subsection{Authenticated Semi-Quantum Direct Communication}

The paper by Luo and Hwang which introduced the Authenticated Semi-Quantum
Direct Communication (ASQDC) protocols - one of which was implemented
in the present work - first introduced usage of Bell states (discussed
below) into ASQKD protocols in order to prevent information leakage,
thus enabling Alice to determine if Eve (a malicious actor) extracted
information from the transmission. The details of how this is accomplished
are given below.

\subsection{Bell States}

Bell states are the four maximally-entangled states that can be produced
using two qubits. When the state of one qubit is measured (reduced
to a classical bit), the state of the other qubit is instantly known
by the party who measured the first. Quantum entanglement - the name
for this phenomenon - is a physical property and is not something
that can be logically circumvented. If, for example, two photons (qubits)
are entangled into a Bell state and one of the photons is sent to
another location, the sender will instantly know the value of the
transmitted photon when they measure the remaining photon. 

\subsection{Cases Implemented}

Five cases are considered: no attacker, Eve impersonates Alice, Eve
impersonates Bob, Eve intercepts and re-sends the message, and Eve
modifies one qubit of the message.

\subsubsection{No Attacker}

In the case where there is no attacker, it is expected that the protocol
will function every time (given that, in the simulation, there are
no environmental factors). Alice first computes the hash $h(m)$ of
the message $m$ and concatenates them into $M=m||h(m)$. The constraints
on the hash function $h$ are that it must output a hash equal in
length to the message and it must never produce collisions. Given
these constraints, the authors opted to simply flip the value of each
bit from the message in the hash. While this would not be secure in
the wild, Eve will only be brute-forcing the keys in this simulation.
Furthermore, it ensures that the protocol can be used to send any
length of message without hash collisions. Next, Alice will produce
a sequence of Bell-EPR pairs $S$ such that if $M[i]$=0 then $S[i]=\left|\phi^{+}\right\rangle $.
Otherwise, $S[i]=\left|\psi^{-}\right\rangle $. Alice then generates
a random sequence of Bell-EPR pairs $C$ in the basis $\left\{ \left|\phi^{+}\right\rangle ,\left|\psi^{-}\right\rangle \right\} $
with the same length as $S$, labeling the first half $C_{b}$ and
the second half $C_{a}$. Next, Alice tensors $S$ and $C$, producing
$SC_{ba}$. $SC_{b}$ is then randomly shuffled using key $K_{1}$
and sent to Bob as $Q$. Alice retains $C_{a}$.

Upon reception of $SC_{b}$, Bob reorders $S^{\prime}C_{b}^{\prime}$
using $K_{1}$ and then perfroms a Z-basis measurement on $S^{\prime}$,
obtaining the measurement result $MR_{B}$, from which Bob can computer
the message and hash $M^{\prime}=m^{\prime}||h\left(m\right)^{\prime}$.
Bob then computes $h\left(m^{\prime}\right)$ and compares it to $h\left(m\right)^{\prime}$
and, if they are not equal, Bob and Alice terminate the protocol.

If $h\left(m^{\prime}\right)=h\left(m\right)^{\prime}$, Bob reorders
$C_{b}^{\prime}$ using $K_{2}$ and reflects the result, $C_{b}^{\prime\prime}$
back to Alice.

Alice then applies $K_{2}$ to $C_{b}^{\prime\prime}$, obtaining
$C_{b}^{\prime\prime\prime}$ and then compares $C_{b}^{\prime\prime\prime}C_{a}$
with $C$ and, if they are equal, believes that the protocol worked
properly.

\subsubsection{Impersonating Alice}

Eve can communicate with Bob, pretending to be Alice, by simply taking
on the role of Alice in the protocol described above, using Eve has
the requisite quantum computing capabilities described above. The
only difference is that Eve needs to guess which keys $K_{1},K_{2}$
to use.

\subsubsection{Impersonating Bob}

Likewise, Eve can impersonate Bob by receiving the message from Alice
and using the part of the protocol normally used by Bob. As above,
Eve will need to guess the keys $K_{1},K_{2}$.

\subsubsection{Intercept-and-Resend Attack}

In this attack, Eve first impersonates Bob to Alice and then imperonsates
Alice to Bob, essentially conducting a MITM attack. The point of this
attack, from Eve's perspective, is to determine the content of the
message by measuring what she receives from Alice. Again, as above,
Eve must guess the keys $K_{1},K_{2}$.

\subsubsection{1-Qubit Modification Attack}

In this attack, Eve performs a MITM attack, similar to above, but
rather than measuring the message to determine its content, she modifies
one of the qubits and then sends the resulting message to Bob.

\section{Simulation of Protocol with and without Attacks}

\subsection{Summary}

The results achieved mostly reflected the predictions of the creators
of the protocol. The protocol always worked in the absence of an attacker,
as expected. In the cases involving Eve either impersonating Alice
or Bob, the protocol could execute successfully only in the cases
where Eve guesses both keys correctly, or where both keys are wrong
but in such a way that in combination they right each other.

\subsection{Security Analysis}

\subsubsection{Impersonating Alice Attack}

Since Eve does not know the key $K_{1}$, she must guess it. If $K_{1}$
is $n$ bits long, then the probability of Eve guessing the correct
key is $\frac{n}{2^{n}}$. In the event that she does guess $K_{1}$
correctly, she will successfully impersonate Alice, thus fooling Bob.
In the present work, the length of the key used during simulation
was $n=12$, resulting in a probability $\frac{12}{2^{12}}=\frac{3}{1024}$
that Eve correctly guessed the key. 

\subsubsection{Impersonating Bob Attack}

Since Eve does not know the key 

\subsubsection{Intercept and Resend Attack}

\subsubsection{1-Qubit Modification Attack}

\section{Evaluation}

The measured results do support the claims of the randomized-ASQDC
protocol. The implementation tested in the current work behaved as
expected in the face of five attacker cases: no attacker, impersonation
of Alice, impersonation of Bob, intercept-and-resend attack, and 1-qubit
modification attack. 

\pgfplotstableread[row sep=\\,col sep=&]{
    outcome                                 & Expected & Result \\
    No Attacker: Success                    &    10000 &  10000 \\
    Impersonate Alice: Detected             &     7500 &   7464 \\
    Impersonate Bob: Detected               &     9961 &   9891 \\
    Intercept-and-resend: Detected by Bob   &     7500 &   7441 \\
    Intercept-and-resend: Detected by Alice &     9961 &   9954 \\
    Intercept-and-resend: Detected          &     9990 &   9985 \\
    1-Qubit Modification: Detected by Bob   &     6667 &   6678 \\
    1-Qubit Modification: Detected by Alice &     3333 &   3322 \\
    1-Qubit Modification: Detected          &    10000 &  10000 \\
}\mydata

\begin{figure}
    \vspace{1em}
    \caption{Expected vs. simulated results after 10,000 simulations with $n=16$}
    \label{fig:resultsGraph}
    \vspace{1em}
    \begin{tikzpicture}
        \begin{axis}
            [
                xbar,
                ylabel = {Outcome},
                xlabel = {\# occurrences},
                xlabel style = {
                    yshift = 1em,
                },
                bar shift = {-0.4*(\numplotsofactualtype*\pgfplotbarwidth + (\numplotsofactualtype-1)*(2pt)) - (.5+\plotnumofactualtype)*\pgfplotbarwidth - \plotnumofactualtype*(2pt)},
                symbolic y coords = {
                    No Attacker: Success,
                    Impersonate Alice: Detected,
                    Impersonate Bob: Detected,
                    Intercept-and-resend: Detected by Bob,
                    Intercept-and-resend: Detected by Alice,
                    Intercept-and-resend: Detected,
                    1-Qubit Modification: Detected by Bob,
                    1-Qubit Modification: Detected by Alice,
                    1-Qubit Modification: Detected,
                },
                tick label style = {
                    font = \small,
                },
                y tick label style = {
                    anchor = west,
                    xshift = 0.1cm,
                    yshift = 0,
                },
                y dir = reverse,
                xtick = {0, 10000},
                scaled x ticks = false,
                ytick = data,
                xmin = 0,
                xmax = 10000,
                nodes near coords,
                bar width = .3cm,
                height = 15cm,
                width = 8cm,
                legend style = {
                    at={(0.5, 1)},
                    anchor = north,
                    legend columns = -1,
                },
            ]
            \addplot table[y=outcome, x=Expected]{\mydata};
            \addplot table[y=outcome, x=Result]{\mydata};
            \legend{Expected\,\,\,\,, Result}
        \end{axis}
    \end{tikzpicture}
\end{figure}

\section{Conclusion}

In summary, the current work tested a Matlab with QIT implementation
of the randomization-based ASQDC protocol in five sets of circumstances:
no attacker, impersonation of Alice, impersonation of Bob, intercept-and-resend
attack, and 1-qubit modification attack. The measured results were
then compared against the theoretical values, supporting the claims
of the ASQDC protocol authors.

This is relevant to quantum communications research as such a protocol
may be more feasible than fully-quantum protocols in a variety of
circumstances in the short-to-medium term. Furthermore, it was shown
that the protocol is, in fact, secure against four major categories
of attack.

Future work would include testing the protocol against additional
attacks, such as n-qubit modification attacks. It would also be of
benefit to test the protocol on computers which are capable of handling
larger values of $n$ (longer messages) in order to demonstrate empirically
that the security does in fact scale with message length as predicted
by theory.

\begin{thebibliography}{00}
\bibitem{b1} Y.-P. Luo, T. Hwang, ``Authenticated semi-quantum direct communication protocols using Bell states,'' Quantum Information Processing, vol. 15, no. 2, pp. 957--958, November 2015.
\end{thebibliography}
\vspace{12pt}

\appendix
\section{Matlab Code}
All Matlab code for the project is attached below. The {\ttfamily Execute.m} script was used to generate the simulated results in Fig. \ref{fig:resultsGraph}.
{\ttfamily numExecutions} on line 5 is the number of simulations to run for each type of attack, which can be modified from 10,000.

\onecolumn
\definecolor{MatlabComment}{rgb}{0, 0.6, 0}
\lstset {
    language=Matlab,
    frame=single,
    breaklines=true,
    showstringspaces=false,
    basicstyle=\footnotesize\ttfamily,
    commentstyle=\color{MatlabComment},
    keywordstyle=\color{blue},
    stringstyle=\color{magenta},
}
\lstinputlisting[title=Alice.m]{"../Alice.m"}
\lstinputlisting[title=Bob.m]{"../Bob.m"}
\lstinputlisting[title=Eve.m]{"../Eve.m"}
\lstinputlisting[title=Execute.m]{"../Execute.m"}
\lstinputlisting[title=helper.m]{"../helper.m"}
\lstinputlisting[title=utilities.m]{"../utilities.m"}

\end{document}
